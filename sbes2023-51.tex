\documentclass[sigconf]{acmart}
%\documentclass[sigconf,anonymous,review]{acmart}

\usepackage{enumitem}
\usepackage{listings}

\copyrightyear{2023}
\acmYear{2023}
\setcopyright{acmlicensed}\acmConference[SBES 2023]{XXXVII Brazilian Symposium on Software Engineering}{September 25--29, 2023}{Campo Grande, Brazil}
\acmBooktitle{XXXVII Brazilian Symposium on Software Engineering (SBES 2023), September 25--29, 2023, Campo Grande, Brazil}
\acmPrice{15.00}
\acmDOI{10.1145/3613372.3614190}
\acmISBN{979-8-4007-0787-2/23/09}

\begin{document}

\title{Teaching and Promoting Engagement with OSS: Yet Another Experience Report}
%\title[ER]{Teaching and Promoting Engagement with OSS:\\An Experience Report}
%%
%% The "author" command and its associated commands are used to define
%% the authors and their affiliations.
%% Of note is the shared affiliation of the first two authors, and the
%% "authornote" and "authornotemark" commands
%% used to denote shared contribution to the research.
\author{Christina von Flach}
\authornote{Both authors contributed equally to this research.}
\email{flach@ufba.br,dfeitosa@ufba.br}
\orcid{1234-5678-9012}
\author{Daniela Feitosa}
\authornotemark[1]
%\email{dfeitosa@ufba.br}
\affiliation{%
  \institution{Federal University of Bahia, Institute of Computing}
  \streetaddress{Milton Santos Ave}
  \city{Salvador}
  \state{Bahia}
  \country{Brazil}
  \postcode{40170-110}
}

\renewcommand{\shortauthors}{Flach, C. and Feitosa, D.}

%----------------------------------------%
\begin{abstract}
%, artefacts and practices of the open-source development model 
OSS-based learning refers to using open source software (OSS) and their sociotechnical practices in the pedagogical context. Several educators reported its benefits and barriers within different contexts, goals, and areas of knowledge, bringing evidence that it is a feasible approach to address Software Engineering Education challenges. 
Some appealing factors to adopting OSS-based learning are the availability of the software source code and workflows, access to the OSS community, and information about its development and evolution, which may improve the learning of SE concepts and practices and foster students' engagement in real-world projects.
Observing students as they work towards hard and soft skills, understanding, engaging with, and eventually contributing to an OSS project is a rewarding part of the routine of educators who have recognized the benefits of OSS-based learning and adopted it in their classes.
We report our experience after delivering a ``hands-on'' course to introduce graduate students to OSS projects and their sociotechnical practices. 
Some graduate students were higher education instructors in other institutions. We present the course design and details of its execution, followed by a reflection based on students' feedback and our perceptions of gains and pains.
Overall, students valued the course and were highly motivated to explore OSS, especially those who worked as educators. Most of the feedback concerning the course methodology was positive, 
but some students requested more information about the weekly lesson plans in advance.
We hope this experience report helps to demystify OSS, inspire educators to adopt OSS projects in their courses,  and foster instructors' and students' engagement with OSS projects.

\end{abstract}

%Open Source Software (OSS) projects allow users to run, study, modify, and redistribute the software with few eventual restrictions. 

%----------------------------------------%
%% The code below is generated by the tool at http://dl.acm.org/ccs.cfm.
%% Please copy and paste the code instead of the example below.
%%
\begin{CCSXML}
<ccs2012>
<concept>
<concept_id>10003456.10003457.10003527.10003531.10003751</concept_id>
<concept_desc>Social and professional topics~Software engineering education</concept_desc>
<concept_significance>500</concept_significance>
</concept>
<concept>
<concept_id>10011007</concept_id>
<concept_desc>Software and its engineering</concept_desc>
<concept_significance>500</concept_significance>
</concept>
<concept>
<concept_id>10011007.10011074.10011134.10003559</concept_id>
<concept_desc>Software and its engineering~Open source model</concept_desc>
<concept_significance>500</concept_significance>
</concept>
<concept>
<concept_id>10002944.10011122.10002945</concept_id>
<concept_desc>General and reference~Surveys and overviews</concept_desc>
<concept_significance>300</concept_significance>
</concept>
</ccs2012>
\end{CCSXML}

\ccsdesc[500]{Social and professional topics~Software engineering education}
\ccsdesc[500]{Software and its engineering}
\ccsdesc[500]{Software and its engineering~Open source model}
\ccsdesc[300]{General and reference~Surveys and overviews}
%----------------------------------------%

%----------------------------------------%
%% Keywords. The author(s) should pick words that accurately describe
%% the work being presented. Separate the keywords with commas.
\keywords{Open source software, active learning, project-based learning, OSS-based learning, software engineering education.}
%----------------------------------------%
 
 \maketitle

%----------------------------------------%
\section{Introduction}
%----------------------------------------%

Software Engineering Education (SEE) is challenging, even for the most experienced instructors\footnote{In this paper, the term \textit{instructor} denotes ``educator or teacher in a higher education course,'' and we will use it throughout the text.}. 
The need to reconcile theory and practice~\cite{Liguo:Book:2014,feliciano:2016,garousi2019closing, nascimento:oss:2019}, using ``real-world'' scenarios to stimulate the improvement of technical and soft skills 
is demanding for Software Engineering (SE) and Computer Science Education in general~\cite{spinellis:acm:2021}.

Open Source Software (OSS) has become prevalent in the software industry. Moreover, some companies expect potential employees to have an active GitHub profile that showcases their OSS contributions~\cite{robles:2019}. 
OSS allows users to study, run, modify, and redistribute the software under licenses\footnote{\url{https://opensource.org/osd/}}.
The availability of source code and the access to its community and development workflows are appealing factors for adopting OSS in SEE~\cite{Bishop:2016, nascimento:fie:2018, DBLP:conf/icse/0001FSSM19}.

OSS-based learning\footnote{OSS-based learning is a general term coined by and used internally in our research group.}
refers to using realistic scenarios from OSS and related socio-technical practices as pedagogical objects.
In practice, students encounter and engage with OSS projects
used as the primary medium for learning.
OSS-based learning is not limited to developing students' technical skills but also addresses general soft skills such as problem-solving, critical-thinking, teamwork, and communication, which are often overlooked~\cite{garousi2019closing}. 
It may encourage students to contribute to OSS projects, with the first contact and following steps smoothly and carefully mediated by the instructors~\cite{spinellis:acm:2021}.
%
Confident instructors who adopted OSS-based learning as part of undergraduate and graduate courses,  either using instructor-oriented OSS projects~\cite{tos:2023, nascimento:fie:2018, DBLP:conf/sbes/SilvaBTC19} or OSS projects for which they had no control~\cite{pinto:cseet:2017, DBLP:conf/sbes/FerreiraS0SM18, DBLP:conf/icse/0001FSSM19} reported the positive outcomes and barriers they faced -- some of which are similar to those from adopting OSS in general~\cite{stein:2014}.

Instructors can experience the benefits of OSS-based learning by carefully addressing related challenges, such as selecting appropriate OSS projects for pedagogical use~\cite{smith:2014} and getting familiar with its environment, practices, and rules.
%
However, course planning and delivery are labor intensive and may be intimidating. Instructors may need to familiarize themselves with new technologies and practices and reconcile these with the theory and pedagogical concerns~\cite{Silva:Santos:Flach:2023}.
%
In this context, experience reports are an essential source of information.


%Hypotheses Matrix: Studies have revealed that user training positively influence the adoption of OSS.
%https://arxiv.org/pdf/1901.00750.pdf

\begin{comment}
\citet{DBLP:conf/sbes/BritoSCNB18} surveyed the literature for studies on the use of OSS projects in SEE and found that 69\% of the primary studies published between 2014 and 2018 were \textit{experience reports}.
We performed an \textit{ad-hoc} literature review and searched explicitly for \textit{experience reports published between 2019 and 2022}. \citet{dorodchi2019teaching} reported 
an undergraduate course for SE with emphasis on open source,
teamwork, and modeling. The results
reveal that students are satisfied with the course model for
teamwork, however, not as much satisfied with the open source
activities which is in-line with literature due to the challenges
of open source on installation, configuration, and running on
different hardware with different operating systems.
Their observations and initial findings indicated that students enjoyed the open source challenges and demonstrated professional competency after the course, with a positive impact of open source and teamwork on students.
\citet{ryan:icse:seet:2022} reported their experience on a course in which students should learn how to read the code of an existing, large-scale OS system to become contributing members of its community.
\end{comment}

In this paper, we report our latest experience with OSS-based learning in a SE course entitled \textit{Topics in Software Engineering} (TSE). % $<anonymous>$.
%Práticas e Reflexões sobre Contribuição em Projetos de Software Livre.
The course design aimed to reconcile theory with practice (evolving software in social coding platforms) while planning to address previously reported barriers of OSS-based learning~\cite{nascimento:fie:2018, nascimento:oss:2019, pinto:cseet:2017, DBLP:conf/sbes/FerreiraS0SM18, DBLP:conf/icse/0001FSSM19}.  
The course did not intend to replace a mandatory SE course but to prepare students for future project courses, including a SE course.
We expect that the experience and lessons shared in our report may be helpful to educators willing to deploy relatively novel and challenging teaching practices such as OSS-based learning~\cite{Silva:Santos:Flach:2023}.
  
This experience report is structured as follows. 
First we provide some background on OSS and OSS-based learning (Section~\ref{sec:background}). Then 
we present the course design, including learning outcomes and course structure (Section~\ref{sec:course_design}) followed by details about the course implementation (Section~\ref{sec:weeks}).
We present students' feedback about the course (Section~\ref{sec:students_feedback}) and confront it with our own experience to discuss lessons learned and challenges for future course implementations (Section~\ref{sec:reflections}) and provide some recommendations (Section~\ref{sec:final:remarks}).

%Writing an experience report is an endeavor task.
%We report our experience following the format used by \citet{ryan:icse:seet:2022} to report their experience on a course in which students should learn how to read the code of an existing, large-scale system to become contributing members of its community. 
%We report our experience with the course entitled $<anonymous>$ as follows.

%----------------------------------------%
\section{Background}
\label{sec:background}
%----------------------------------------%
%\subsection{Open source software development}

\paragraph{Open Source Software Development (OSSD)}
The OSSD workflow is often driven by means of open repositories, hosted by version control systems which allow multiple versions of the same software to coexist. 
%
Open repositories share several kinds of assets such as source code, tests, reports, and workflows. 
While the repository is often publicly available for reading access, writing access is restricted to a limited group of developers selected by meritocracy. 

Collaboration among peers is frequent and enabled via shared access to source code and multiple communication channels. 
%
Code review (``given enough eyeballs, all bugs are shallow'') fosters enhanced software quality by means of sharing, collaboration, and peer review and can be applied to other assets. 
Automated testing increases reliability and maintainability, and promotes agility in the development of new features. 
The use of standard data formats and interfaces facilitates the integration with other systems and discourages vendor lock-in~\cite{flach:sbc:2021}. 

Frequent and continuous documentation is a recommended practice to keep user guides, manuals, and other relevant documents updated with respect to the latest version of the software. 
Finally,
OSS projects seamlessly support artifact availability, peer collaboration, workflow transparency, reuse and reliability~\cite{flach:sbc:2021}.
Due to its nature, practices, and communities, we consider  OSS projects suitable pedagogical objects for SEE.

%\subsection{OSS in SEE}
\paragraph{OSS-based Learning.}
We define OSS-based learning as an specialization of project-based, active learning methods.
%
OSS-based learning is mostly about teaching and learning socio-technical practices from OSS projects.
Instructors want students to read and understand code, assess its quality, find and resolve bugs, test, among other activities that can be aligned with most of the learning objectives of SE courses. 
OSS-based learning also values social practices, developing soft skills, and the experience of joining a real, possibly diverse community, and becoming a contributor.

%Therefore, we consider that common practices used by OSS communities can be good practices to be used in SEE.

%We performed an ad-hoc literature review on related course work to find courses that focuses on experience with open-source systems.
\begin{comment}
\paragraph{Experience Reports.}
\citet{DBLP:conf/sbes/BritoSCNB18} surveyed the literature for studies on the use of OSS projects in SEE and found that 69\% of the primary studies published between 2014 and 2018 were \textit{experience reports}~\citep{id17882, id0135, id5343, id5353, id17796, id5357, id1088, id4503, id4663, id4811, id17805, id17830,id17845,id18433, id5329, id5335, id0089, id0115, id17800, id0106, id4966, id18359, id5328}.
%
%\citet{DBLP:conf/sbes/FerreiraS0SM18} presented...
%
We performed an \textit{ad-hoc} literature review and searched specifically for \textit{experience reports published between 2019 and 2022}. The closest to our work is that of \citet{dorodchi2019teaching}. Their observations and initial findings indicated that students enjoy the open source challenges and demonstrate professional competency after this course, with a positive impact of open source and team work on students.

%Some aspects of the learning experience in most of these courses were simulated rather than real \cite{holmes2018dimensions}.
%
%Research paper: 
%\citet{DBLP:conf/icse/0001FSSM19} conducted 21 semi-structured interviews with students in five different Brazilian universities. 
%They also did an analysis of commits performed in the repositories that these students contributed to. Students’ contributions varied concerning both complexity (measured by the number of additions, deletions, and edited files) and diversity (measured regarding the different programming languages used). Among the benefits, students reported improving their technical skills and their self-confidence. 
%Finally, some students found \textbf{extremely important for instructors’ being involved with open source initiatives} (extra-classroom).

\citet{ryan:icse:seet:2022} presented an experience report on a course that addressed the objective of learning how to read the code of an existing, large-scale system to become an effective contributing member of its community. 
We report our experience on a course designed to address socio-technical practices and activities used in OSS projects and possibly encourage students to become a contributing member of the community.
    
\end{comment}

%----------------------------------------%
\section{Course Design}
\label{sec:course_design}
%----------------------------------------%

%----------------------------------------%
\subsection{Context} \label{sec:context}
%----------------------------------------%

%The use of OSSD practices is widespread ...and SEE ...
%A decade ago, we devised a course that adopted the approach of contributing to open-source projects as part of undergraduate and graduate courses. 
%We delivered this course four times but did not share our experience with the SEE community. 
% Internally, we referred to such an approach as ``OSS-based learning''.
%This time we planned and documented the course design more carefully.
%We planned more hand-ons activities during the class, and introduced the use of journals and invited talks.
% The course involved the instructor, a teaching assistant, with two graduate students performing the role of observers.

%<anonymous>$ 
TSE is an elective course that uses OSS-based learning to reconcile theory with practice by introducing students to SE socio-technical practices used in OSS development. The audience comprises undergraduate and graduate students in Computer Science. 
% Rebuttal R1.Q1.
Undergraduate students should be able to handle the course easily since the discipline has, as a prerequisite, Software Engineering, a mandatory 60-hour course taught in the seventh semester. 

Over the years, the course design has evolved into the version presented in this experience report.
%
In its latest version, the course had two instructors: a SE researcher with experience in OSS (the first author of this report) and \texttt{TA-1}, a teaching assistant proficient in OSS\footnote{Ruby on Rails developer with experience in training (Rails, Linux, and OSS practices and tools) and women in technology initiatives (Rails Girls).} (the second author). One MSc student, \texttt{TA-2}, helped the instructors in mentoring students during hands-on activities. 

We adopted virtual, synchronous classes\footnote{Supported by the institutional Google Meet.} since we delivered the course during the covid-19 pandemic. 
We had a 3-hour meeting once a week on Fridays, structured as a ``stand-up briefing'', lecture or invited seminar in the first half, a 10-minute coffee break, and hands-on activities or presentation of students' reports in the second half.  

We used GitHub as a social coding and learning platform to mediate our discussions and support the course activities.
%
We created a GitHub organization to support the course, two teams 
(one for the instructors and the other for students) and a template repository with the resources that students would need during the course (textbooks, papers, websites, software, and support services). 

%----------------------------------------%
\subsection{Learning Outcomes} \label{sec:learning_outcomes}
%----------------------------------------%
%Start by identifying the learning outcomes for the course.
%List expectations around course objectives (or learning outcomes).

The course's goal was to support students in becoming familiar with at least one OSS project, helping them to increase confidence in modern technical practices and to improve social skills.  
% What do you want students to achieve by the end of the course? 
By the end of the course, students should have a comprehensive view of OSS concepts and familiarity with an OSS project and its socio-technical aspects. Moreover, they should experience OSS in a natural environment, using a social coding platform, and eventually becoming a contributing member of its community. 
%
% What skills, knowledge, or attitudes do you want them to develop?
We mapped the goals above to concrete learning outcomes (1--4) that governed the design of the course. 
Students should be able to:
\begin{enumerate}[label=\textbf{LO\arabic*}]
\item Understand Open Source Software concepts and practices.
\item Characterize OSS projects regarding its features, history and context, newcomers' support and social conventions, socio-technical practices and deliverables.
\item Select OSS projects to work with based on their domain and other characteristics.
\item Understand aspects related contributing to OSS projects.
\end{enumerate}

The contribution to the selected OSS required at least sending a patch or 
making a pull request (PR), taking into account the project's standards and social conventions~\cite{ryan:icse:seet:2022}.
%Whether those standards are explicitly documented or part of an informal culture that has emerged and has to be learned informally.

The course's goal was not to train the students with specific social coding platforms or tools, for instance, GitHub.
% Our goal was not to make them effectively contribute to internal or closed projects.
We expect them to get familiar with the Git version control system\footnote{\url{https://git-scm.com/}} -- it is a necessary path to 
desmistify and onboard students to the OSS world.
Therefore,  we asked students to create or use an existing GitHub account, 
to get started with GitHub\footnote{\url{https://docs.github.com/en/get-started/quickstart/hello-world}} and use it regularly during the term.

Besides technical skills, the course also fostered students' communication skills
by requiring oral presentations and open written reports and, stimulating their interaction with the community of the selected OSS.


%Journal, GitHub discussions, Students' Reports.

%List assessments, schedules, readings, different types of assignments, rules & policies, grading, etc. Or add them to Github and mention it.
 
\subsection{Topics}
\label{syllabus}

%\textbf{Course objectives:} Start by identifying the learning outcomes for the course. What do you want students to achieve by the end of the course? What skills, knowledge, or attitudes do you want them to develop?

%\textbf{Course content:} Once you have identified the learning outcomes, determine what topics, concepts, and skills you need to cover in the course. Make sure that the content aligns with the learning objectives.

The course addressed topics such as: 
%practices and group reflections on contributing to OSS projects; encouraging student participation in OSS projects; 
OSS basics; technical, social, and legal aspects of OSS projects; social organization and management in OSS projects;  contributions to OSS projects and interactions with their communities; and quality of OSS projects. Such topics are detailed as follows:
\begin{itemize}
%\tightlist
\item
    Open source software basics: GNU, Free Software Foundation (FSF), Open Source Initiative (OSI), ``The Cathedral and Bazaar''.
\item
   Legal Aspects of OSS: Why Share Software? the concept of
   ``commons'', large-scale benefits. How to share? Legislation
   about copyright, copyleft, common, recommended licenses (GPL, Apache,
   MIT, etc.).
\item
   Social organization of OSS: How developers gain access and
   reputation in OSS projects. Meritocracy concepts and
   ``do''-cracy. Communication in OSS projects (mailing list, IRC, web
   forums, blog aggregators etc).
\item
   OSS project management: Version control systems (SVN, Git), bug tracking systems (Bugzilla, Redmine, etc), and related practices.
\item
   Contributions to OSS projects: How to interact with the community
   open source software; translations, documentation, bug reporting, preparation and submission of ``patches'' (new features or bug fixes) or pull requests (PR).
\item
   Quality assurance in OSS projects: Coding styles,
   receiving and reviewing ``patches'', automatic tests,
   continuous build/integration, bug triage.
\end{itemize}

\subsection{Course Structure} \label{sec:structure}

The course structure included lectures, recommended readings, invited talks, discussions, assignments with hands-on activities, students' written reports, and a workshop.
Students should select OSS projects hosted by the GitHub platform to get acquainted and contribute. Section~\ref{sec:weeks} presents the course design in practice.

\subsubsection{Lectures}
We settled on a course design
that included lectures followed by students' hands-on activities and discussions mediated by the instructors.
Table~\ref{tab:lectures} presents the course plan.

\subsubsection{Readings} \label{leituras}

Recommended readings, all openly available online, 
were provided at least a week before each lecture to enable students to prepare themselves for practical activities and discussions. Some required readings included: 
%(Section~\ref{leituras}).
material from FSF\footnote{\url{https://www.fsf.org/}} and OSI\footnote{\url{https://opensource.org/}}, and chapters 1--5 from the book
``Open Sources: Voices from the Revolution''~\cite{dibona1999sources}.


The \texttt{opensource.com} initiative\footnote{\url{https://opensource.com/}}, supported by RedHat\footnote{\url{https://www.redhat.com/}}, hosts articles and ebooks, and describes them as a \textit{chronicle of the history of the open source movement in the eyes of some prolific ``actors'' and authors}. Some available readings are\footnote{\url{https://opensource.com/resources/ebooks}}: 
``Free Software, Free Society'' by Richard M. Stallman, 
``Just for Fun'' by Linus Torvalds and David Diamond,
``Free for All'' by Peter Wayner, 
``Free as in Freedom'' by Sam Williams and
``The Cathedral and the Bazaar'' by Eric S. Raymond~\cite{ESR:Cathedral}.

For project characterization, we recommended the book ``Object-Oriented Reengineering Patterns''~\cite{demeyer2008}.
For other hands-on activities we recommended ``Producing Open Source Software'' by Karl Fogel~\cite{ProducingOSS}, 
``Open Sources'' by Chris DiBona and Sam Ockman~\cite{dibona1999sources}, ``The Open Source Way (TOSW)''\footnote{\url{https://www.theopensourceway.org/book/}},
``The Open Source Way 2.0''\footnote{\url{https://www.theopensourceway.org/the_open_source_way-guidebook-2.0.html}},
and ``Conventional Commits''\footnote{\url{https://www.conventionalcommits.org/en/v1.0.0-beta.2/\#summary}}.

\subsubsection{Invited Talks}

To enrich and bring experience to the classroom, we invite Brazilian OSS contributors with distinguished roles in the community or extensive experience with OSS to present short talks (30 to 45 minutes) to be followed by students' questions and discussions -- everything recorded with the consent of the participants.

Students should prepare in advance questions on the topic of the talk. 
After the talk, they should ask their questions, participate in discussions,
and write a short critical summary.

\subsubsection{Journals}

Students should write a journal/diary to document their activities during the course. We provided a minimal template that required a date,  a list of finished tasks, and a list of things they planned to do.
%(Listing~1).
%
Students should be encouraged to provide additional information and record their experiences, opinions, difficulties, and impressions about the activities.

%\begin{lstlisting}[caption=Journal entry template.]
%
%# DATE
%
%## DONE
%+ Activity 1
%+ Activity 2 
%+ ...
%
%## TODO
%+ Activity 3
%+ Activity 4 
%+ ...
%
%\end{lstlisting}

We expected that writing individual journals would foster students' engagement in the course routine, their commitment to activities, and incremental recording of positive or negative experiences in engaging with the OSS project.
Journals and their entries would also allow tracking the student's learning process and contribution to projects, identifying problems, and encouraging discussions during weekly meetings. 

\subsubsection{Assignments}

%\paragraph{\textit{OSS characterization.}}

Two assignments require the characterization of an OSS project: 
\begin{enumerate}[label=\textbf{A\arabic*}]
    \item \textit{Work with OSS pre-selected by the instructors.}
    Students should select an OSS project from a set of projects from the domain of \textit{personal journals}, with features for keeping daily records of events and experiences pre-selected by the instructors.
    \item \textit{Work with OSS project selected by you.} 
    Students should select (with our mentoring) an OSS project hosted by the GitHub platform to get acquainted, identify issues, and contribute. Students should explain to their colleagues the rationale for their selection and decisions.
\end{enumerate}

Characterizing an OSS project involves performing hands-on activities to get acquainted with the OSS and writing a report.
Table~\ref{tab:activities} presents hands-on activities concerned with OSS characterization.

\begin{table}[htb]
    \caption{Hands-on Activities to characterize OSS projects.} \label{tab:activities}
    \begin{tabular}{c|l}
       \textbf{Activity} & \textbf{Description} \\
    \hline
        \textbf{C1} & What are the main features of the project? \\
        \textbf{C2} & How is the project documentation? \\
        \textbf{C3} & Any concerns about internationalization?  \\
        \textbf{C4} & How does the project use the issue tracker?  \\
        \textbf{C5} & Was it worth it?  \\
    \hline
    \end{tabular}
\end{table}


%When we present the activity, we discuss how the information added would be essential to generate history, providing data for reflection and analysis of the course organization and its content.
Appendix~\ref{app:reconhecimento-tecnico} presents a report template for characterizing OSS projects.
For both assignments, students should record their activities and thoughts in their diaries (journals),  participate in virtual ``stand-up'' meetings and share their progress and experiences with their colleagues.

\subsubsection{Assessment}

During the term, we assessed  the students' work through their participation in the course activities and the quality of the assignments' deliverables, for instance:
\begin{itemize}
    \item Participation in synchronous activities, such as invited talks, hands-on tasks, and discussions (20\%);
    \item Individual questions and summaries prepared for invited talks, and deliverables within the first assignment \textit{Work with OSS selected by the instructors} (30\%); and
    \item Written reports, journal entries, artifacts within the second assignment \textit{Work with an OSS project selected by you}  and oral presentations during the workshop (50\%).
\end{itemize}

% Rebuttal R1.Q2.
We defined a rubric with four levels for the assignments: Needs Improvement (less than 5), Satisfactory [5-7), Good [7-9), and Accomplished [9-10]. We wanted to give frequent constructive feedback without discouraging students. When necessary, we postponed the deadlines. The students performed well with good grades, with an average of 8.71 and a standard deviation of 0.65. 


% Bonus for pull request(s) accepted or returned with requests for improvement.

%----------------------------------------%

\section{The Course in Practice}
\label{sec:weeks}
%, possibly due to the number of activities of a class that was optative in their curriculum.

Table~\ref{tab:lectures} presents the course plan organized in 17 weekly meetings, highlighting the primary topics of each lecture, in-class activities associated with those topics, invited talks, and individual assignments for each upcoming week.
%

Fifteen students started the course, but only 12 completed the course.
Along these 17 weeks, we introduced students to OSS concepts and practices, 
provided guidelines for the first contact and 
encouraged them to perform primary contributions to OSS projects.
We also coordinated the frequent sharing of individual progress 
and organized the students' workshop to present their reports, 
discuss gains and pains, and exchange feedback.


\begin{table*}[htb]
\centering
\caption{Week-by-week lectures.} \label{tab:lectures}
\begin{tabular}{c|p{6.5cm}|p{4.25cm}|p{4.5cm}}
\hline
\textbf{Lect.} & \textbf{Class topic} &
\textbf{In-class Activity} & \textbf{Homework} \\ \hline 
1 &
Course presentation 
& Discussion and profiling survey
& Readings
\\
2 & 
Open source software basics 
& Discussion & Readings
\\
3 & 
Contributing to OSS projects 
& Demonstration (TA-1) & Assignment A1
\\
4 &
Project Selection
& Demonstration & Assignment A2
\\
5 &
Getting Familiar & Demonstration & A1: activity C1
\\
6 &
Legal Aspects about FLOSS & Talk: Nelson Lago (IME-USP)  & Reading assignments and questions.
%\href{keynotes/talks.md\#nelson-lago}{Nelson Lago} 
\\
7 &
Contributing (1): Documentation & Hands-on & A1: activity C2
\\
8 &
Contributing (2) and (3) & Hands-on & A1: activities C3 and C4
\\
9 &
Mid-Term: Students & Presentation & A1: activity C5
\\
10 &
Social Aspects of FLOSS & Talk: Antonio Terceiro (Debian)
& Reading assignments and questions.
%\href{keynotes/talks.md\#antonio-terceiro}{Antonio Terceiro} 
\\
11 &
Tasks, questions and discussion & Hands-on & A2 tasks
\\
12 &
Project Management in FLOSS: DevOps & Talk: Paulo Meirelles (IME-USP) e Leonardo Leite (SERPRO) 
& Reading assignments and questions.
%\href{keynotes/talks.md\#paulo-meirelles}{Prof.~Paulo Meirelles (UFABC/USP)} e \href{keynotes/talks.md\#leonardo-leite}{Leonardo Leite (USP)} 
\\
13 &
Bug Triage & Demonstration (TA-2) & A2 tasks
%\href{keynotes/talks.md\#nelson-lago}{Nelson Lago} 
\\
14 &
Issues in FLOSS Projects & Talk: Rodrigo Rocha Gomes e Souza (UFBA) 
& Reading assignments and questions\\
%\href{keynotes/talks.md\#rodrigo-rocha-gomes-e-souza}{Prof.~Rodrigo Rocha Gomes e Souza (UFBA)} 
15 &
Bug Resolution & Hands-on &  A2 tasks 
%\href{keynotes/talks.md\#nelson-lago}{Nelson Lago} 
\\
16 & 
Open Source Software Ecosystems: The KDE 
& Talk: Sandro Andrade (IFBA) 
& Reading assignments and questions.
%\href{keynotes/talks.md\#sandro-andrade}{Prof.~Sandro Andrade (IFBA)} 
\\ 
17 &
Students' Workshop & Oral presentations & Students' reports
\\
\hline
\end{tabular}
\end{table*}



The journaling activity started in the third week of the course, 
and students documented their progress and thoughts in the following weeks. 
Some students were prolific and wrote ten or more journal entries. 
The minimum number of entries was two, and the maximum was 19. 
The mean was 6.41 entries per student. 
Listings 1 and 2 present interesting excerpts of students' journals.


\begin{lstlisting}[caption=Example of Journal Entry.]
# Eddy's Diary
## Oct/07/21
I'm more lost than blind in a shootout!!!

I found out how to update the Fork and, 
through it, commit!!!! Fantastic...
I got excited again...

I was able to contribute with documentation 
via Weblate to several projects. 
I started with TheAlgorithms, but the tool 
suggests contributing to other projects. 
I made translation suggestions for 
Portuguese and English :-)

I made my first contribution of code 
and documentation.

## Oct/08/21
End my pull request for code change and 
documentation addition. 
Waiting for reviewers.
\end{lstlisting}

\begin{lstlisting}[caption=Another example of Journal Entry.]
# Angels' Diary
## Sep/24/2021 Requested to join project
https://github.com/doxygen/doxygen

## Oct/07/21 Joined project participation
lists.

## Oct/08/21 Project has a specific place 
for opening issues: 
https://github.com/doxygen/doxygen/issues. 
Well organized and categorized by labels.
\end{lstlisting} 


\paragraph{\textbf{Week 1: Course presentation.}}

We presented the course design, goals, expected learning outcomes, activities, resources, recommended readings, and assessment methods.
We asked students to create a GitHub account and take a course on GitHub basics\footnote{\url{https://docs.github.com/en/get-started/quickstart/hello-world}}. 
We also invited them to join a GitHub Organization created to support the course, fork a repository\footnote{\url{https://github.com/mate28-ic-ufba/turma-20212/blob/main/README.md}} to have access to course material and create a branch to push their assignments.
We recommended reading five chapters of the Open Sources - vol.1 book and skimming the FSF and OSI websites.

Then, we asked them to participate in a survey to collect their previous experience with OSS and to sign an agreement term for sharing digital assets and videos for educational purposes.

\paragraph{\textbf{Week 2: Concepts e History.}}

We presented and discussed the following topics:
Open source software basics - GNU, Free Software Foundation (FSF), Open Source Initiative (OSI), ``The Cathedral and Bazaar''. 
%
Invited talks and instructors presented other essential concepts, such as licenses and some OSS practices.

\paragraph{\textbf{Week 3: Working with an OSS project selected by the instructors.}}

Before selecting the individual OSS project to work with, 
the assignment \textit{Work with OSS selected by the instructors} puts students and instructors together to present basic concepts (theory) and discuss them while performing hands-on activities (practice). 

We resorted to small projects to present the practices, learn and balance student skills.
The instructors used criteria based on domain and project size to filter OSS projects that supported  \textit{personal journals}, with features for keeping daily records of events and experiences. Each student should select and incrementally characterize one of the OSS projects (Table~\ref{tab:activities}) and prepare and present a characterization report. Appendix~\ref{app:reconhecimento-tecnico} presents the report template for characterizing the OSS project. 

\paragraph{\textbf{Week 4: Project Selection.}}
Before the assignment \textit{Work with OSS selected by you},
we introduced and discussed the features of three tools that address project selection:
(1)~Explore Github\footnote{\url{https://github.com/explore}};
(2)~Ohloh (source code at GitHub) and  Openhub\footnote{\url{https://www.openhub.net}};
(3)~FlossSearch.edu (academic prototype for project selection in the context of SEE)\footnote{\url{http://191.252.92.63/flosssearch/welcome/about}}.
We also recommended reading a paper about selecting OSS projects to contribute with
\cite{10.1145/2591028.2600812}.
%\cite{terceiro:sugarloafplop:2012}.
We mentored them to carefully skim over documentation before project selection, focusing on project size, welcome messages, and good first issues.

\paragraph{\textbf{Week 5: Getting Familiar}}

Students informed their selected OSS projects.
We recommended reading the third section of a paper that documented ``Involvement Patterns'' to get acquainted with the selected OSS project before making any contributions
\cite{10.1145/2591028.2600812} 
and the third chapter of the OORP book (``First Contact''\footnote{\url{http://scg.unibe.ch/download/oorp/}}).

\paragraph{\textbf{Weeks 7, 8, 11, 13, 15: Working with an OSS project selected by the student.}}
Students selected a single OSS project as their practice playground during hands-on activities and after-class homework. 
Within each week or two weeks of introducing new OSS practices, 
we posted and explained the activities to perform, 
moving from the first contact with the project and its characterization 
to eventually making contributions in the context of C2, C3 and C4 (see Table~\ref{tab:activities}).

Instructors encouraged students to make small contributions as early as possible to address documentation, translation, and minor tracked issues.
Table~\ref{tab:oss_projects} presents the  OSS projects selected by the students.

\begin{table}[ht]
    \centering
    \caption{Selected OSS Projects (https://github.com)}
    \label{tab:oss_projects}
    \begin{tabular}{l|p{2cm}|p{3.75cm}}
\hline
Id  & Project  & URL 
\\ \hline
P1 & opendrinks & /alfg/opendrinks
\\
P2 & PHP-MySQL-ecommerce-website & /hammadshahir/PHP-MySQL-ecommerce-website
\\
P3 & Common Voice & /common-voice/common-voice
\\
P4 & Zulip Terminal & /zulip/zulip-terminal
\\
P5 & SpotBugs & /spotbugs/spotbugs
\\
P6 & Django b. & /zostera/django-bootstrap4
\\
P7 & Ethereum & /ethereum/ethereum-org-website
\\
P8 & TheAlgorithms & /TheAlgorithms/C
\\
P9 & Doxygen & /doxygen/doxygen
\\
P10 & MuseScore & /musescore/MuseScore
\\ 
P11 & Beekeeper Studio & /beekeeper-studio/beekeeper-studio
\\
\hline
    \end{tabular}
\end{table}


\paragraph{\textbf{Weeks 6,10,12,14,16: Invited Talks}}

We invited active people with different roles in the OSS community 
to share their expertise and present talks about 
``Software Licenses'', ``Social Aspects of OSS'', ``OSS project management and DevOps'', ``Issues in Free Software Projects'', and ``Free Software Ecosystems (KDE)''.

The presentations introduced new topics, followed by students' questions and exciting discussions. 
For each talk, the students had to prepare questions on the talk subject, 
watch it and participate, ask their questions, and write a critical summary.

\paragraph{\textbf{Week 17: Students' Workshop}}

The students should deliver a technical report after accomplishing the second assignment and before the workshop date.
We provided a template to structure the oral presentations to cover a minimum set of essential topics. Students were encouraged to read each other reports and participate actively in the workshop.

We planned to execute the students' workshop in one virtual synchronous class with several individual presentations. Still, due to extensive discussions, we had to attend an additional meeting in the same week to summarize the scheduled presentations.

%----------------------------------------%
\section{Student's Feedback}
\label{sec:students_feedback}
%----------------------------------------%

At the beginning of the course, we asked students to complete a survey to profile their experience with OSS.
Fifteen students participated in this survey, but we consider the answers of the twelve students that completed the term as valid.
%
Most of the participants were graduate students (91.6\%). 
The first survey identified that, \textit{before joining the course}:
%29 years of age, or older (75\%), (ii)~most students worked in the SE field (75\%) -- in the industry (50\%) or educational institutions (25\%),
(i)~most graduate students did not have any prior experience contributing to an OSS project, 
(ii)~three graduate students were higher-education instructors but hardly knew OSS in practice, 
and (iii)~only one participant was an undergraduate student but with some experience with OSS.
% (vi)~only one student did not have a formal education in Computing Science but rather in Performing Arts, Music.

At the end of the course, we asked students to complete a second survey that probed their overall course experience and allowed them to provide open-ended comments with feedback on its positive and negative points, and suggestions for improvement.
%
The survey comprised five sections: Presentation and Consent Form, Participant Profile, Isolation Context, Self-evaluation
and Experience Report. 
In our experience report, we present only the relevant data from ``Participant Profile'', ``Self-evaluation'' and ``Experience Report'' to support our reflections (Section~\ref{sec:reflections}). 
Appendix~\ref{app:questionario} presents the instrument used for the feedback survey.

\subsection{Feedback Survey}

\subsubsection{Participant Profile}

Twelve students participated in the second survey and all of them signed the consent form.
%
Table~\ref{tab:population} presents demographic information about the students. 
Participation in the survey was anonymous and we assigned an identification number (S1--S12) to each respondent.
%
In a nutshell: (i)~the participants self-declared their gender as \textit{male} (58.3\%) or \textit{female} (41.7\%);
(ii)~most participants were 29 years of age or older (75\%); (iii)~when asked whether they worked professionally in the area (software engineering or computer science), three declared to be instructors, five to be developers,  one worked as a project manager, and two students reported no previous experience in the area; and, (iv)~concerning their education level, all of them were students at UFBA, with eleven graduate students (seven MSc students, one Ph.D. student, and three Ph.D. candidates), 
and only one (senior) undergraduate student.

\begin{table}[thb]
    \centering
        \caption{Students' Profile}
    \label{tab:population}
    \begin{tabular}{c|c|c|c|c|c}
    \hline
        \textbf{Id} & \textbf{Gen} & \textbf{Age} & \textbf{Experience} & \textbf{Education} & \textbf{Project} \\
    \hline
        S1 & M & +29 & Educator & MSc & P3 \\
    \hline
        S2 & F & +29 & No &  MSc & P1 \\
    \hline
        S3 & M & 26--29 & Developer & MSc & P2 \\
    \hline
        S4 & M & +29 & Educator & PhD Candidate & P11 \\
    \hline
        S5 & F & +29 & Developer &  PhD Candidate & P8 \\
    \hline
        S6 & M & +29 & Developer &  PhD Student & P7 \\
    \hline
        S7 & F & +29 & Student & MSc & P6 \\
    \hline
        S8 & F & +29 & Manager & MSc & P10 \\
    \hline
        S9 & M & 22--25 & Developer & Bach & P5 \\
    \hline
        S10 & M & +29 & Educator & MSc & P3 \\
    \hline
        S11 & F & 26--29 & No & PhD Candidate & P9 \\
    \hline
        S12 & M & +29 & Developer & MSc & P4 \\
    \hline
    \end{tabular}
\end{table}

\subsubsection{Self-evaluation}

Students evaluated themselves with respect to eight aspects (Table~\ref{tab:likert_2}). Most of the students were positive and agreed with the phrases.
For the record, one student declared that 
\textit{I marked ``I totally disagree'' in some items in the self-assessment, not because the course did not provide such experiences, but because I have already experienced them.}
%
%Nonetheless, we are not going to detail and discuss self-evaluation results in this experience report.

\begin{table*}[bt]
    \centering
    \caption{Self-evaluation (TA-Totally Agree; A-Agree; N-Neutral; D-Disagree; TD-Totally Disagree).}
    \label{tab:likert_2}
    \begin{tabular}{l|p{13cm}|c|c|c|c|c}
    \hline
    & \textbf{Question} & \textbf{TA} & \textbf{A} & \textbf{N} & \textbf{D} & \textbf{TD}\\
    \hline
    1 & I can grasp the complexity and difficulties in building models of a software project in public repositories &8 &4 &&& \\
    2 & I can identify the content discussed in the classroom in a real-world, medium-sized software &10 &2 &&& \\
    3 & I can perform activities similar to those performed during the course in another software project &9 &3 &&& \\
    4 & The experience contributed to the improvement of my professional performance (current or future) &11 &&&& 1\\
    5 & The experience contributed to the development of the proactivity skill &10 &1 &&& 1\\
    6 & The experience contributed to the development of problem solving skills &9 &2 &&& 1 \\
    7 & The experience contributed to the development of communication skills &10 &&1 && 1\\
    8 & The experience helped me to have contact with code developed by third parties &10 & 2&&& \\
    \hline
    \end{tabular}
\end{table*}

%Eight students could grasp the complexity and difficulties in building models of a software project in public repositories.

%S1, S2, S5, S5, S7, S8, S9, S10 totally agreed.
%S4, S6, S11 and S12 agreed.

%----------------------------------------------------
%\subsection{Experience report}
%Thematization of benefits, challenges, recommendations, etc.
\subsubsection{Experience report.}

In the fifth section of the survey, the students answered eight open-ended questions: 
\begin{enumerate}
\item How was your experience in the OSS project selection process (positive and negative points)?
\item How was your experience carrying out the selected project's characterization (positive and negative points)?
\item How was your experience carrying out the practical activity on bug resolution (positive and negative points)?
\item Describe your experience in building the ``journal'' (positive and negative points) 
\item Describe your experience with the OSS community (commits, community feedback, etc.)
\item Did the course meet your expectations? Please comment.
\item Were the strategies used in the classes during the semester adequate? Please comment on positive points and points for improvement. Suggestions are welcome.
\item General comments (optional).
\end{enumerate}

In this experience report we present a subset of the survey open-ended questions and their answers, with no intention of providing a comprehensive qualitative analysis. As mentioned before, analysis and detailed discussions about the results of the second survey study are ongoing work.

The questions selected are those concerned with the course quality:
\textit{(6)~Did the course meet your expectations? Please comment.}, 
\textit{(7)~Were the strategies used in the classes during the semester adequate? Please comment on positive points and points for improvement. Suggestions are welcome.}, and
\textit{(8)~General comments (optional)}.

%About the course

\subsection{Students' Reported Experience}

\subsubsection{Did the course meet your expectations?}
%%Table~\ref{tab:expectations} presents ...
\begin{table}[htb]
    \caption{Did the course meet your expectations?}
    \label{tab:expectations}
    \begin{tabular}{c|p{7cm}}
    \hline
    Part. & Opinion \\ 
    \hline
       S1  &  Yes. One of the main reasons for me to join the course was to learn how to contribute to free software. 
       \\
       S2 & Yes, I really enjoyed it. They were very dynamic and flexible.
       \\
       S3 & My objective in the course was to acquire more knowledge about OSS projects. In this sense, it met my expectations.
       \\
       S7 & Yes, but I imagined there would be more hands-on activities. \\
       S9 & Yes. The course managed to meet what I expected from it, which was to contribute to an open source project and get me started in the open source universe.
       \\
       S11 & Yes. The course surprised me in a very positive way. I overcame my fear of interacting and proposing solutions in OSS projects.
       \\
       S12  & Yes. However, if the project was internal to the class, it would have been better.\\
       \hline
    \end{tabular}
\end{table}

%Table~\ref{tab:expectations} presents excerpts of the students' answers for this question.
Most of the students reported as their basic expectation \textit{to acquire (more) knowledge about OSS projects}.
%

%% Reviewer 2, paper weakness: Refining the presentations of the survey results (i.e., avoiding overusing quotations)
The feedback regarding the expectations met during the course was positive. Students expressed satisfaction in learning to contribute and overcoming fears of interacting and proposing solutions in FLOSS projects. Additionally, they appreciated the exposure to new technologies, best practices and the flow of activities in these projects. The course openess, without a predefined theme, enabled participants to explore and engage in collaborative software development. Overall, the course successfully met the expectations of the participants, promoting a context where they could actively engage and contribute to the open-source community.


\subsubsection{Were the strategies used during the semester's classes adequate?}

Most of the students provided positive feedback concerning the course methodology. Many of them mentioned points for improvement and made objective suggestions.


%% Reviewer 2, paper weakness: Refining the presentations of the survey results (i.e., avoiding overusing quotations)
\textit{\textbf{Students's Gains.}}
In general, the course received positive feedback regarding the methodology applied, with students finding it very good and effective. The strategies used, including talks and practical applications of knowledge were adequate and valuable in promoting active learning. The flexibility in choosing types of contributions motivated the students to engage in areas aligned with their personal interests. The use of journals to record activities proved beneficial, with some students planning to maintain this practice beyond the course. The implementation of active learning techniques, continuous deliveries and talks were also appreciated and contributes to a rich learning experience.

While most of the feedback concerning the course methodology was positive, its delivery and execution were not widely perceived as such by all students.

%% Reviewer 2, paper weakness: Refining the presentations of the survey results (i.e., avoiding overusing quotations)
\textit{\textbf{Students's Pains.}}
One area of concern was the limited interactions and support provided, especially for students with little experience in programming and Git. Discussions over asynchronous channels were not properly addressed. Some participants felt that the learning curve associated with GitHub limited the development of the initial activities and led to discomfort. Additionally, one student reported the lack of hands-on activities and a more practical approach. Other student was not satisfied with the organization of classes, citing a lack of pre-planning and uncertainty about the course structure. Communication difficulties were also pointed out.


%% Reviewer 2, paper weakness: Refining the presentations of the survey results (i.e., avoiding overusing quotations)
\textit{\textbf{Suggestions.}}
A student suggested combining a public GitHub project with internal control project to promote better assimilation of concepts and practical knowledge. Students mentioned that the course should develop further the concepts and practices of internationalization. Additionally, a student suggested a dedicated class or lecture on project management, highlighting the potential to provide a better comprehension of the activities. Incorporating industry experts as invited talks was also suggested for insights into other possible real-world problems. Lastly, they requested more regular presentations of the students' activities to improve experience and knowledge sharing.

%---------------------------------%
\subsubsection{General Comments}

In the last question of the survey, we asked students to freely provide general comments about the course. In this report, we took an \textit{ad-hoc}, naive strategy and  grouped some comments under a few  topics that may deserve a more systematic analysis and detailed discussion as future work.


%% Reviewer 2, paper weakness: Refining the presentations of the survey results (i.e., avoiding overusing quotations)
\textit{\textbf{Demystifing OSS.}}
Some students mentioned that the greatest legacy of the discipline was empowering them to actively participate in the open source community, making them feel safe and motivated to start projects on GitHub and to participate in others. Also, beyond the programming skills, they realized there are different ways to contribute to software projects, requiring only goodwill.

%% Reviewer 2, paper weakness: Refining the presentations of the survey results (i.e., avoiding overusing quotations)
\textit{\textbf{Paradigm Shift.}} As reported by the students, the course changed their view of free software and introduced a new paradigm for thinking about the development of ideas.  The course has also helped them with valuable insights that they anticipate applying not only to their research work but also to future software projects.

%% Reviewer 2, paper weakness: Refining the presentations of the survey results (i.e., avoiding overusing quotations)
\textit{\textbf{Education and Career.}}
The students expressed their appreciation for the course's benefits and the knowledge gained. One student emphasized that it would contribute to his future career achievements. Another participant shared his enthusiasm to apply the acquired knowledge in his master's research and embrace OSS in his software projects. A student regretted not having been exposed previously to the course's topics during his academic life.


%% Reviewer 2, paper weakness: Refining the presentations of the survey results (i.e., avoiding overusing quotations)
\textit{\textbf{Nucleation.}}
Working with instructors and encouraging them to adopt OSS practices is one of the main contributions of the course to SEE. The course's participants that were also instructors, reported they started requiring their students to use Git as part of their academic activities. The knowledge acquired enabled them to adopt the practices in data structures lectures, in which they now encourage students to test code, open and address issues, and use tags. Some reported having limited experience with Git and not taking advantage of what the control version system offers. Finally, a participant stated that the course not only awakened a desire to do the same for his students but also served as a source of inspiration for his colleagues.


\textit{\textbf{Use of internal projects.}}
Instructors have to decide on a control  strategy to be used in the course~\cite{nascimento:oss:2019}:  one or more
\textit{internal OSS projects} to their institution or course, for which they have total control, or  \textit{external OSS projects} with ``no control'',  and that depend on their core members to accept or not a student contribution.
One of the students stated that ``if the projects were internal to the class, it would have been better.''
%
However, another student had a different opinion; he expressed a positive sentiment regarding the decision to work with an external OSS project. The student considered such an experience more meaningful because, in his view, the challenge is to understand how this environment works in practice and look for ways to join the community and contribute.

\textit{\textbf{Students' acknowledgements.}}
The students expressed gratitude to the course instructors, monitors, and everyone involved in creating the activities and supporting them. 

%----------------------------------------%
%\section{Reflections}
%\label{sec:reflections}
%----------------------------------------%

\section{Instructors' Reflections} \label{sec:reflections}

We present some reflections on the course design and execution, and the students' feedback.

% Rebuttal R2.Q1 
\begin{description}
\item \textit{Course design.} 
From related work~\cite{pinto:cseet:2017,DBLP:conf/sbes/FerreiraS0SM18,DBLP:conf/icse/0001FSSM19} and our previous experience~\cite{2015:CSE:nascimento,nascimento:fie:2018,nascimento:oss:2019},
we were primarily concerned with warm-up activities, project selection, interaction with the community, and the course time constraints.
%
We resorted to small ``journal'' projects to present OSS concepts and practices. While selecting their individual OSS project, we mentored the students to carefully skim over documentation, focusing on project size, welcome messages, and good first issues. 
Communication with students and support for their interaction with the OSS community still deserve our attention in the future. 

\item \textit{Git-Paralysis.} 
In previous courses, we noticed that some students experienced a mix of fear and rejection of the mental model for cloning, branching, opening pull requests, or pushing their work into a public repository. 
Students freeze or even drop the course. 
Therefore, we committed ourselves to addressing issues related to demystifying GitHub, OSS development, minimizing prejudice against OSS, and stimulating and valuing any contribution.

Instructors set the course expectations early, including the requirement of sharing knowledge and contributing to the selected OSS projects, and twelve students accepted these challenges.
Hands-on activities with careful guidance coped well with the symptoms and eased the illness.

\item \textit{Journaling.}
The activity of journaling started in the third week of the course. 
Some students were prolific and wrote ten or more journal entries. 
There is room to stimulate the practice of journaling with benefits for both the instructors and the students. Journal entries may also be helpful to researchers involved with OSS-based learning.

\item \textit{Project selection.} 
The selection of an alternative OSS project to contribute with, although not recommended, was not forbidden. 
One student complained about the project he selected (not responsive and open to contribution in practice), and we supported him in choosing another one.
Working with the same project was not forbidden either.
Two students worked on the Common Voice project and performed different contributions. They noticed they had selected the same OSS project by chance while using its issue tracker. 

\item \textit{Contributions to OSS.}
A significant achievement recognized by the participants is that some students managed to perform submissions of pull requests and enjoyed the experience. However, some needed more confidence to practice social coding.
We highlight that contributions are not exclusively about developing features and fixing bugs. In our course, we recommend students start with translations and reviews.

\item \textit{Growing soft skills.} 
%Software Engineering is a sociotechnical discipline~\cite{Li:Ko:Zhu:2015}.
OSS projects require skills that span both hard and soft skills~\cite{liang:skills:2022}. 
%
While performing the course activities, students had the opportunity to use soft skills such as problem-solving, collaboration, communication, and critical thinking, among others. Future work will investigate the role of OSS-based learning in developing students' soft skills.

\item \textit{Dissemination.}
\citet{ryan:icse:seet:2022} states that \textit{students that learn and apply a principled approach to an unfamiliar system, integrated with the project's social context, and contribute a piece of code to the system, may have an important advantage in securing a future position}.

The three graduate students who worked as higher education instructors
recognized the relevance of OSS and were motivated to contribute to OSS projects and to adopt OSS practices and resources in their classes. 

\end{description}

\section{Final Remarks} \label{sec:final:remarks}

The success of adopting OSS-based learning depends on educators who challenge themselves, embrace new pedagogical strategies, reconcile them with the course syllabus, learn from hands-on experience, and spread the word.
%
We hope that this experience report reaches them.

Below, we provide some recommendations 
from the trenches for educators interested in OSS-based learning.

\begin{enumerate}
\item Hands-on activities are the core of the course: recruit at least one teaching assistant with experience in OSS development.
\item Promote open communication: define communication channels early and use them effectively.
\item Decide on a control strategy to use early in the course (internal or external projects).
\item Plan to be flexible. 
\begin{itemize}
    \item Plan and communicate the syllabus early, leaving room for last-minute, on-the-fly modifications.
    \item Survey the students' knowledge and previous experience with OSS and adjust the course plan, if necessary.
\end{itemize}
\item Select suitable OSS projects.
\begin{itemize}
    \item Define criteria, filter and select a set of small OSS projects to use as internal learning objects.
    \item Double-check the OSS project for welcome messages and guidelines for newcomers.
\end{itemize}
\item Mentor students in selecting an OSS project to contribute.
\item Share early and share often: share journal entries and students' progress, and promote frequent discussions.
\item Invite keynotes to share their expertise and present talks about software licenses, continuous SE, social aspects of OSS, among other topics.
\end{enumerate}

%----------------------------------------%

\section*{Artefact Availability}

Course design, students' diaries (journals), activities' outcomes, and reports are available at \url{https://github.com/mate28-ic-ufba/turma-20212.git}. 
The surveys' instruments and data are available at \url{https://doi.org/10.5281/zenodo.8252922}.

\begin{acks}
Two researchers played the role of observers with no interaction with the instructors and the students during the course. We thank them for the co-design of the survey instruments used to profile students and collect their feedback, and for allowing us to use and share their data partially.
\end{acks}

\balance

\bibliographystyle{ACM-Reference-Format}
\bibliography{sbes2023-edu}

\appendix

\section{Technical Characterization Form}
\label{app:reconhecimento-tecnico}
The template for the initial characterization report on the selected OSS project.

\subsection{Part I - Project description}
\label{part-i-description-of-the-project}

\begin{enumerate}
\def\labelumi{\arabic{enumi}.}
\item
   Name
\item
   Web site
\item
   Description
\end{enumerate}

\subsection{Parte II - Initial characterization}
\label{parte-ii---reconhecimento-inicial}

\begin{enumerate}
\def\labelenumi{\arabic{enumi}.}
\setcounter{enumi}{3}
\item
  How long has the project existed? Briefly describe the history of the project.
\item
  What is the project license?
\item
  What is the project size (lines of code, classes)?
\item
  Has the project size grown in recent versions?
\item
  Is there recent activity on the project?
\item
  Is there collaboration from companies on the project?
\item
  What are the technologies used?
\end{enumerate}

\subsection{Parte III - Issues characterization}
\label{parte-iii---identificauxe7uxe3o-de-tarefas}

\begin{enumerate}
\def\labelenumi{\arabic{enumi}.}
\setcounter{enumi}{10}
\item
  Where is the source code repository? Which version control tool is used?
\item
  What kind of bug repository is used and where is it located?
\item
  Is there documentation to help new contributors?
\item
  Are there any bugs that are marked as easy to fix?
\item
  Does the source code contain issues lists?
\item
  Are there any comments containing TODO, FIXME, etc in the source code? What kind of need are placed in these comments?
\end{enumerate}

\subsection{Parte IV - Technical characterization}\label{parte-iv---caracterizauxe7uxe3o-tuxe9cnica}

\begin{enumerate}
\def\labelenumi{\arabic{enumi}.}
\setcounter{enumi}{17}
\item
   How many developers participated in the project in the last 6 months? And from the beginning?
\item
   What is the version release policy?
\item
   Does the project have an automated test suite?
\item
   Are there parts of the source code that are problematic? Justify.
\item
   Does the code contain comments? Are there parts that need further comment?
\item
   Can design patterns and architectural patterns be identified from the code? Which ones did you identify?
\item
   How was the project source code obtained?
\item
   What did you need to install on your system to make the project
   compile/run? Was the documentation provided by the project sufficient?
\end{enumerate}


\section{Survey instrument for Students Feedback}
\label{app:questionario}
% Survey Instrument
%Everything should be available at: ZENODO.

The survey instrument to gather the students' feedback about the course is presented below. 

\subsection{Presentation}
\textit{Dear Student,}

\textit{This survey aims to evaluate the course MATE28-Topics in Software Engineering IV, which dealt with Practices and Reflections on Contribution in Free/Libre Open Source Software Projects (FLOSS) in 2021.2.}

\textit{Completing the questionnaire may take 10 to 15 minutes. Your responses will be treated completely anonymously.}
 
\textit{Thanks for staying with us until the end of the course. Now, your participation is very important in this evaluation and self-assessment process!}

\textit{The SEED.br Group Team.}

\subsection{Consent Form} 
(Ommitted)

\subsection{Participant Profile}

\begin{enumerate}
    \item Gender
    \item Age
    \item Education level
    \item Professional experience
    \item General comments about experience (open-ended question).
\end{enumerate}

\subsection{Isolation Context}
(COVID-19. Ommitted)

\subsection{Self-evaluation}

\begin{enumerate}
    \item Selected OSS project: name and GitHub URL (open-ended)
    \item I can grasp the complexity and difficulties in building models of a software project in public repositories (Likert scale)
    \item I can identify the content discussed in the classroom in a real-world, medium-sized software. (Likert scale)
    \item I can perform activities similar to those performed during the course in another software project. (Likert scale)
    \item The experience contributed to the improvement of my professional performance (current or future). (Likert scale)
    \item The experience contributed to the development of the proactivity skill. (Likert scale)
    \item The experience contributed to the development of problem solving skills. (Likert scale)
    \item The experience contributed to the development of communication skills. (Likert scale)
    \item The experience helped me to have contact with code developed by third parties. (Likert scale)
\end{enumerate}

\subsection{Experience Report}

Open-ended questions.

\begin{enumerate}
\item How was your experience in the OSS project selection process (positive and negative points)?

\item How was your experience in carrying out the characterization of the selected project (positive and negative points)?

\item How was your experience in carrying out the practical activity on bug resolution (positive and negative points)?

\item Describe your experience in building the ``journal'' (positive and negative points) 

\item Describe your experience with the OSS community (commits, community feedback, etc.)

\item Did the course meet your expectations? Please comment.

\item Were the strategies used in the classes during the semester adequate? Please comment on positive points and points for improvement. Suggestions are welcome.

\item General comments (at will).

\end{enumerate}


\end{document}
